\documentclass{article}

\begin{document}

\title{\bf Workshop Proposal: International Workshop on Logical
       Frameworks and Meta-Languages: Theory and Practice (LFMTP)}

\author{Fr\'ed\'eric Blanqui (Inria, France) and Giselle Reis (CMU Qatar)}

\date{19 June 2017}

\maketitle

\section{Scientific content}

\subsection{Description}

  Logical frameworks and meta-languages form a common substrate for
  representing, implementing, and reasoning about a wide variety of
  deductive systems of interest in logic and computer science. Their
  design and implementation on the one hand and their use in reasoning
  tasks ranging from the correctness of software to the properties of
  formal computational systems on the other hand have been the focus
  of considerable research over the last two decades. This workshop
  will bring together designers, implementors, and practitioners to
  discuss various aspects impinging on the structure and utility of
  logical frameworks, including the treatment of variable binding,
  inductive and co-inductive reasoning techniques and the expressivity
  and lucidity of the reasoning process.

  LFMTP will provide researchers a forum to present state-of-the-art 
  techniques and discuss progress in areas such as the following:
\begin{itemize}
\item Encoding and reasoning about the meta-theory of programming
    languages and related formally specified systems.
\item Theoretical and practical issues concerning the treatment of
    variable binding, especially the representation of, and reasoning
    about, datatypes defined from binding signatures.
\item Logical treatments of inductive and co-inductive definitions and
    associated reasoning techniques.
\item New theory contributions: canonical and substructural frameworks,
    contextual frameworks, proof-theoretic foundations supporting
    binders, functional programming over logical frameworks, homotopy
    type theory.  
\item Applications of logical frameworks, e.g., in proof-carrying
    architectures such as proof-carrying authorization.
\item Techniques for programming with binders in functional programming
    languages such as Haskell, OCaml, or Agda and logic programming
    languages such as $\lambda$-Prolog or $\alpha$-Prolog.
\end{itemize}
  
\subsection{History}

  The proposed workshop continues the series of workshops on logical
  frameworks and meta-languages that started in 1999. It began with
  LFM and MERLIN, which were interleaved until 2005. These two
  workshops were merged in 2006 to form LFMTP, which has has grown
  into one of the main venues for presenting results on type theories
  and logical frameworks.

\subsubsection{LFM: International Workshop on Logical Frameworks and Meta-languages (1999 - 2004)}

\begin{itemize}
\item LFM'99, Paris, France, affiliated with PPDP and PLI, organized by
     Amy Felty (now University of Ottawa)

\item LFM'00, Santa Barbara, USA, affiliated with LICS, organized by
     Joelle Despeyroux (INRIA)

\item LFM'02, Copenhagen, Denmark, affiliated with LICS at FLOC,
     organized by Frank Pfenning (Carnegie Mellon University).

\item LFM'04, Cork, Ireland, affiliated with IJCAR, organized by
     Carsten Schürmann (Yale University)
\end{itemize}
   
\subsubsection{MERLIN: International Workshop on MEchanized Reasoning about Languages with variable BInding (2001 - 2005)}

\begin{itemize}
\item MERLIN 2001, Siena, Italy, affiliated with IJCAR, organized by
     Roy L. Crole, Simon J. Ambler, and Alberto Momigliano (then at
     University of Leicester, UK)

\item MERLIN 2003, Uppsala, Sweden, affiliated with PLI, organized by
     Alberto Momigliano (then at University of Leicester, UK) and
     Marino Miculan (University of Udine, Italy)

\item MERLIN 2005, Tallinn, Estonia, affiliated with ICFP, organized by
     Alberto Momigliano (then at University of Edinburgh), Ivan
     Scagnetto (Universiy of Udine), and Alwen Tiu (INRIA Lorraine)
\end{itemize}
   
\subsubsection{LFMTP: International Workshop on Logical Frameworks and Meta-Languages: Theory and Practice (2006 - present)}

\begin{itemize}
\item LFMTP'06, Seattle, USA, affiliated with LICS and IJCAR at FLoC,
     organized by Brigitte Pientka (McGill University) and Alberto
     Momigliano (University of Edinburgh)

\item LFMTP'07, Bremen, Germany, affiliated with CADE, organized by
     Carsten Schürmann (IT University of Copenhagen) and Brigitte
     Pientka (McGill University)

\item LFMTP'08, Pittsburgh, USA, affiliated with LICS, organized by
     Andreas Abel (LMU) and Christian Urban (TUM)

\item LFMTP'09, Montreal, Canada, affiliated with CADE, organized by
     Amy Felty (University of Ottawa) and James Cheney (University of
     Edinburgh)

\item LFMTP'10, Edinburgh, Scotland, affiliated with LICS at FLoC,
     organized by Karl Crary (Carnegie Mellon University) and Marino
     Miculan (University of Udine).

\item LFMTP'11, Nijmegen, The Netherlands, affiliated with ITP,
     organized by Herman Geuvers (Radboud University Nijmegen) and
     Gopalan Nadathur (University of Minnesota).

\item LFMTP'12, Copenhagen, Denmark, affiliated with ICFP, organized
     by Adam Chlipala (MIT) and Carsten Schuermann (ITU)

\item LFMTP'13, Boston, USA, affiliated with ICFP, organized by Alberto
     Momigliano (University of Milan), Brigitte Pientka (McGill
     University), and Randy Pollack (Harvard University)

\item LFMTP'14, Vienna, Austria, affiliated with CSL-LICS and IJCAR as
     part of FLoC and VSL'14, organized by Amy Felty (University of
     Ottawa) and Brigitte Pientka (McGill University)

\item LFMTP'15, Berlin, Germany, affiliated with CADE-25, organized by
     Iliano Cervesato (Carnegie Mellon University) and Kaustuv
     Chaudhuri (INRIA)

\item LFMTP'16, Porto, Portugal, affiliated with FSCD'16, organized by
     Gilles Dowek (ENS Cachan) and Dan Licata (Wesleyan University)

\item LFMTP'17, Oxford, UK, affiliated with FSCD'17, organized by
     Florian Rabes (Jacobs University, Bremen) and Marino Miculan
     (University of Udine)
\end{itemize}
   
\section{Organization}

\subsection{Workshop chairs}

\begin{itemize}
\item Fr\'ed\'eric Blanqui (Inria), frederic.blanqui@inria.fr
\item Giselle Reis (CMU-Qatar), giselle@cmu.edu
\end{itemize}

\subsection{Steering Committee}

\begin{itemize}
\item Iliano Cervesato (Carnegie Mellon University)
\item Kaustuv Chaudhuri (Inria)
\item Adam Chlipala (MIT)
\item Amy Felty (University of Ottawa)
\item Alberto Momigliano (University of Milan)
\item Brigitte Pientka (McGill University, chair)
\item Carsten Schürmann (IT University of Copenhagen) 
\end{itemize}

\subsection{Potential Program Committee Members (TBD)}

\begin{itemize}
\item Christian Urban (King's College London)
\item Andrej Bauer (University of Ljubljana, Slovenia)
\item Ana Bove (Göteborg University)
\item William Farmer (McMaster University, Canada)
\item Stéphane Graham-Lengrand (CNRS, France)
\item María Alpuente (Catedrática de Universidad, Spain)
\item Aaron Stump (University of Iowa) 
\item Herman Geuvers (Radboud University)
\item Carlos Olarte (UFRN, Brasil)
\item Chantal Keller (LRI, Université Paris-Sud, France)
\item Yuting Wang (Yale University, USA)
\end{itemize}

\subsection{Hosting Conferences}

   Our primary proposed hosting conference is FSCD. We propose
   also to be affiliated with IJCAR. As can be seen by the history,
   LFMTP has been affiliated with many related conferences over the
   years, including these 2 conferences.

\subsection{Estimate of Audience Size}

   LFMTP usually attracts around 30-40 people.

\subsection{Proposed Format}

   The workshop program will be composed of regular papers (formally
   reviewed and published), invited talks, and informal
   work-in-progress talks.

\subsection{Potential Invited Speakers}

   LFMTP and its predecessor workshops have had many invited speakers
   over the years. In 2014, the speakers were Gopalan Nadathur, Jesper
   Bengtson and Edwin Brady; in 2015, Frank Pfenning, Vivek Nigam and
   Marc Lasson; in 2016, Joachim Breitner; in 2017, Andrew Appel and
   James McKinna. We will ask the program committee for suggestions.

   Some initial suggestions include:
\begin{itemize}
\item José Meseguer (University of Illinois, Urbana-Champaign)
\item Dale Miller (Inria)
\item William Farmer (McMaster University, Canada)
\item Michael Kohlhase (FAU Erlangen-Nürnberg)
\item Gilles Dowek (Inria)
\item Vincent Rahli (University of Luxembourg)
\end{itemize}

\subsection{Procedures for Selecting Papers and Participants}

   Submissions will include regular papers of approximately 8 pages
   and work-in-progress papers of approximately 4 pages.  As in the
   past, we will have a program committee of 8-10 people who will
   formally review the papers and help select the invited speakers.
   The topics are typically of interest to attendees of a number of
   the main FLoC conferences, including FSCD, IJCAR, LICS and ITP. As
   in the previous years, the workshop will be publicized through
   mailing lists by repeated calls for papers and participation.

\subsection{Plans for Dissemination}

   Accepted papers will be published electronically as part of the ACM
   International Conference Proceedings Series as has been done in
   several recent years.

\subsection{Duration}

   1 day

\subsection{Preferred period}

Since this workshop is of interest to conferences in both blocks of
FLoC conferences, the preferred period for LFMTP is during the mid
FLoC workshops. As mentioned, we propose FSCD as our primary host
conference, but we would like to be affiliated with IJCAR too, and so
holding the workshop between the two would be best for interested
participants. If this is not possible, then the pre-FLoC period seems
more appropriate.

\end{document}
 