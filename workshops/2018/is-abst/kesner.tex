\documentclass[submission,copyright,creativecommons]{eptcs}
\providecommand{\event}{LFMTP 2018} % Name of the event you are submitting to
\usepackage{breakurl}             % Not needed if you use pdflatex only.
\usepackage{underscore}           % Only needed if you use pdflatex.

\title{A Fresh View of Call-by-Need\\
  {\large (invited talk abstract)}
}
\author{\href{https://www.irif.fr/~kesner/}{Delia Kesner}
  \institute{IRIF, CNRS and Université Paris Diderot}
  \institute{School of Computer Science and Engineering}
  \email{kesner@irif.fr}
}
\def\titlerunning{A Fresh View of Call-by-Need}
\def\authorrunning{D. Kesner}

\begin{document}
\maketitle

\begin{abstract}
  Call-by-need is a lazy evaluation strategy which overwrites an
  argument with its value the first time it is evaluated, thus
  avoiding a costly re-evaluation mechanism. It is used in functional
  programming languages like Haskell and Miranda. In this talk we
  present a fresh view of call-by-need in three different aspects:

  We revisit the completeness theorem relating (weak) call-by-need
  with standard (weak) call-by-name. We relate the syntactical notion
  of (weak) call-by-need with the corresponding semantical notion of
  (weak) neededness, based on the theory of residuals. We extend the
  weak call-by-need strategy, which only computes weak head normal
  forms of closed terms, to a strong call-by-need strategy which
  computes strong normal forms of open terms, a notion of reduction
  which is used in proof assistants like Coq and Agda.

  We conclude our talk by proposing some future research directions.
\end{abstract}

\end{document}
