\documentclass[submission,copyright,creativecommons]{eptcs}
\providecommand{\event}{LFMTP 2018} % Name of the event you are submitting to
\usepackage{breakurl}             % Not needed if you use pdflatex only.
\usepackage{underscore}           % Only needed if you use pdflatex.
\usepackage[dvipsnames]{xcolor}
\usepackage{xparse}

\title{Cubical Computational Type Theory and RedPRL\\
  {\large (invited talk abstract)}
}
\author{\href{http://www.favonia.org/}{Kuen-Bang Hou (Favonia)}
  \institute{Institute for Advanced Study, Princeton}
  \email{favonia@gmail.com}
}
\def\titlerunning{Cubical Computational Type Theory and RedPRL}
\def\authorrunning{Kuen-Bang Hou (Favonia)}

\NewDocumentCommand{\FormatKwd}{m}{\textbf{\textsf{#1}}}
\NewDocumentCommand{\RedPRL}{}{\texorpdfstring{\FormatKwd{{\color{red}Red}PRL}}{RedPRL}}

\begin{document}
\maketitle

\begin{abstract}
  \begin{center}
  \emph{Joint work with Carlo Angiuli, Evan Cavallo, Robert Harper and
  Jonathan Sterling.}
  \end{center}
  \medskip

  Recent research revealed the deep connection between type theory and
  homotopy theory, which inspired a series of new type theories. The
  characteristic new features are univalence and higher inductive
  types, which have led to a fruitful development of synthetic
  homotopy theory. Unfortunately, those new features were originally
  introduced as axioms and disrupted the computational content of type
  theory, which affects their practicality in mechanizing proofs.

  To date, the most promising approach to enjoy new features inspired
  by homotopy theory while retaining computational content is through
  cubical structure. Cohen et al. constructed a type theory based on
  De Morgan cubes with both univalence and higher inductive
  types. Influenced by their work, we also built a type-theoretic
  semantics with all the features, but instead based on Cartesian
  cubes. In addition to using a different cubical structure, our
  semantics is built from computational content directly, following a
  computation-first methodology which itself is interesting.

  \href{http://www.redprl.org/}{\RedPRL{}} is our first proof assistant
  based on the new semantics, inheriting the PRL style pioneered by
  the Nuprl proof assistant, which is also based on the
  computation-first methodology (but without cubical structure). We
  are also developing different proof assistants with distinct proof
  theories, in hope to create the best tools for mechanizing homotopy
  theory and other mathematics.

  In this talk, I will describe how the computation-first methodology
  works, compare two cubical type theories and discuss \RedPRL{}.
\end{abstract}

\end{document}
