\documentclass{article}

\begin{document}

\title{Logical Frameworks and Meta-Languages: Theory and Practice}
\date{2024-07-08}
\maketitle

Logical frameworks and meta-languages form a common substrate for
representing, implementing and reasoning about a wide variety of
deductive systems of interest in logic and computer science. Their
design, implementation and their use in reasoning tasks, ranging from
the correctness of software to the properties of formal systems,
have been the focus of considerable research over the last three decades.

The LFMTP workshop brings together designers, implementors and
practitioners to discuss various aspects impinging on the structure and
utility of logical frameworks, including the treatment of variable
binding, inductive and co-inductive reasoning techniques and the
expressiveness and lucidity of the reasoning process.

The 2024 instance of LFMTP was organized by Florian Rabe and Claudio Sacerdoti Coen in Tallinn, Estonia, as a satellite event of the FSCD conference.
	We are very grateful to the conference organizers for providing the infrastructure and local coordination for the workshop as well as to EPTCS for providing the logistics of publishing these proceedings.

The programme committee additionally contained
\begin{itemize}
\item Mauricio Ayala-Rincón (University of Brasilia)
\item Mario Carneiro (Carnegie Mellon University)
\item Kaustuv Chaudhuri (Inria Saclay)
\item Cyril Cohen (Inria Sophia Antipolis)
\item Alberto Momigliano (University of Milan, Italy)
\item Colin Rothgang (IMDEA, Madrid)
\item Sophie Tourret (Inria Nancy \& Loria)
\item Theo Winterhalter (Inria Saclay)
\end{itemize}
Additionally, Alessio Coltellacci and Chuta Sano provided external reviews.
The editors are very grateful for their thorough analysis of all submissions.

The workshop received 8 submissions, of which 6 were presented at the workshop.
Of these, 2 were work-in-progress presentations, and 4 were accepted for these formal proceedings.
Additionally, Carsten Sch\"urmann of IT University of Copenhagen gave an invited talk on Nominal State Separating Proofs.
\end{document}